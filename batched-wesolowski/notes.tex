\documentclass{article}
\usepackage{graphicx}
\usepackage[margin=0.7in]{geometry}
\usepackage[parfill]{parskip}
\usepackage[utf8]{inputenc}
\usepackage{amsmath,amssymb,amsfonts,amsthm}
\usepackage{csquotes}
\MakeOuterQuote{"}

\begin{document}

Let $G = \mathbb{Z}_M$ be the group of units modulo $M$ where $M = 25195908 \ldots$ is the RSA-2048 challenge number (assumed to be of unknown factorization). Let $G$ be the cyclic group of unknown order generated by a generator $g \in G$.

Consider a toy problem: let $N = p_1 p_2 \ldots p_k$ be a product of a large number of primes. We wish to compute the quantities $g^{N/p_i}$ for each $i$.

The relation of this toy problem to plasma cash using RSA accumulators is as follows: first, allow exponents as in $N = \Pi p_i^{e_i}$, including zero exponents. Then $g$ and $g^N$ are accumulated hash values (``accumulator values'') immediately surrounding (say) 100 blocks. Inclusion and exclusion proofs can be made showing that given $g$ and $g^N$ as ``pbulic inputs'', $p_i$ factors into $N e_i$ times; this is much smaller in size than providing 100 merkle inclusion/exclusion proofs. One example of an inclusion proof is to provide $g$ to the cofactor $N/{p^i_{e_k}}$, i.e. to provide $w = g^{N/{p^i_{e_k}}}$ and have the verifier check that $w^{p_i^{e_k}} = g^N$. For $k$ inclusion proofs, the naive solution involves $k$ modular exponentiations to a large number (the cofactor is almost the same size as $N$ itself); but it is clear that these $k$ modular exponentiations all share a large amount of substructure which we can exploit (indeed, the toy problem turns out to be solvable with $\log k$ modular exponentiations with exponents of size similar to $N$). This is only a toy problem because it doesn't generalize to weseolowski's proof of knowledge of exponent scheme or to exclusion proofs; the ``target'' that one proves knowledges of exponent of is not the same.

Solution to the toy problem: we do some precomputations. Set

\begin{align*}
B_0 &= g^{p_{k/2 + 1} \ldots p_{k}} \\
B_1 &= g^{p_1 \ldots p_{k/2}}
\end{align*}

we treat $B_0$ as ``$g$ raised to the cofactor of the leftmost half of the list of primes'' and $B_1$ as ``$g$ raised to the cofactor of the rightmost half of the lits of primes''. In the next round we compute four $B$-values $B_{00}, B_{01}, B_{10}, B_{11}$, each of which is $g$ raised to the cofactor of a quarter of the list of primes. For e.g., $B_{01}$ is $g$ raised to the cofactor of $p_{k/4} \ldots p_{k/2}$, i.e., to $p_1 \ldots p_{k/4} p_{k/2} \ldots p_k$, which can be calculated as $B_0^{p_{k/4} \ldots p_{k/2}}$. Each level of computation has the same total cost (since modular exponentiation is linear in the exponent size, i.e. linear in the log of the exponent). After $\log k$ such computations, we are done.

\end{document}
